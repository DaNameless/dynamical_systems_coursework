\documentclass[10pt]{article}
\usepackage{NotesTeX}
\usepackage{lipsum}

\title{{\Huge Dynamical Systems and Chaos}\\{\Large{---}}}
\author{Mario Cosenza\footnote{\href{https://google.com/}{\textit{Author Website}}}}

\affiliation{Affiliation of Author}
\emailAdd{mcosenza@yachaytech.edu.ec}
\begin{document}
\maketitle
  \flushbottom
  \newpage
  \pagestyle{fancynotes}
  
  \part{Review of dynamical systems}
  The laws of nature and many natural phenomena can be understood through cause-effect relationships, where the evolution of a 
  system depends on its current state and governing rules. These relationships are expressed in the universal language of 
  mathematics—through equations, functions, and deterministic or stochastic rules. The dynamical systems framework provide a powerful
  framework for modeling such behavior across disciplines, from physics and biology to economics and engineering.  

  Whether studying the motion of planets, the spread of diseases, neural activity, or climate patterns, dynamical systems reveal how 
  complex behavior emerges from simple interactions. By analyzing differential equations, iterative maps, or network dynamics, 
  we uncover patterns, stability, chaos, and predictability in nature. The multidisciplinary nature of dynamical systems highlights 
  the unity of scientific inquiry, demonstrating how mathematical principles underlie diverse real-world phenomena.
  \section{Dynamical Systems}

    \begin{definition}
      A dynamical system can be described as a collection or set of time dependent state variables $\mathbf{S}$
      $$\mathbf{S(t)} = \{ X_1(t), X_2(t), ..., X_n(t) \}$$
      
      that evolves with time according to a \textit{specific known deterministic rule} $\mathbf{f}$, such that
      
      $$S(t) \xrightarrow{\mathbf{f}} S(t + \tau)$$
    \end{definition}
    This \textbf{deterministic} rule $\mathbf{f}$ corresponds to a set of operations or procedures that allows to calculate the state
    of a system at time $t + \tau$ from the knolwdge of its state at time $t$.




  
  \sn{With some additional sidenotes}



  \end{document}
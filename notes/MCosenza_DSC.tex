\documentclass[10pt]{article}
\usepackage{NotesTeX}
\usepackage{lipsum}

\title{{\Huge Dynamical Systems and Chaos}\\{\Large{---}}}
\author{Mario Cosenza\footnote{\href{https://google.com/}{\textit{Author Website}}}}

\affiliation{Affiliation of Author}
\emailAdd{mcosenza@yachaytech.edu.ec}
\begin{document}
\maketitle

  \part*{Preface}
  \begin{flushright}
    {\itshape Compiled and expanded by Rolando S. Sánchez\\
    Student of Professor Mario Cosenza\\
    Yachay Tech University}
  \end{flushright}

  This textbook originates from the lecture notes and course materials developed by Professor Mario Cosenza for his renowned course on 
  \textit{Dynamical Systems and Chaos}. As his student, I have undertaken the task of organizing, refining, and supplementing these 
  invaluable resources to create this comprehensive volume.

  The study of dynamical systems provides a unifying framework for understanding complexity across diverse scientific disciplines. 
  From the microscopic interactions of molecules to the macroscopic behavior of galaxies, from the rhythmic patterns of biological 
  oscillators to the unpredictable fluctuations of financial markets - all can be examined through the lens of nonlinear dynamics. 
  Professor Cosenza's lectures masterfully demonstrated how simple mathematical rules can generate both orderly patterns and chaotic 
  behavior, revealing the deep connections between determinism and unpredictability.

  The core content of this book remains faithful to Professor Cosenza's original lectures, which covered:
  \begin{enumerate}
    \item Introduction to Dynamical Systems
    \item Stability Analysis of Fixed Points
    \item Introduction to Chaos
    \item One-dimensional Maps as Models of Chaotic Systems
    \item Transition to Chaos by Intermittency
    \item Transition to Chaos by Quasiperiodicity
    \item Strange Attractors
    \item Repellers, Transient Chaos, Crises, and Control of Chaos
  \end{enumerate}

  In preparing this text, I have:
  \begin{itemize}
      \item Organized the lecture material into a coherent textbook structure
      \item Added detailed derivations and worked examples
      \item Incorporated modern computational illustrations
      \item Included supplementary exercises and problems
  \end{itemize}

  My aim has been to preserve the intellectual depth and pedagogical clarity of Professor Cosenza's original teachings while making the 
  material more accessible to a broader audience. This book is intended for undergraduate students across physical and mathematical sciences 
  who wish to understand how complexity emerges from simple rules.

  \begin{flushright}
    Rolando S. Sánchez
    \end{flushright}

  \flushbottom
  \newpage
  \pagestyle{fancynotes}
  
  \part{Review of dynamical systems}
  The laws of nature and many natural phenomena can be understood through cause-effect relationships, where the evolution of a 
  system depends on its current state and governing rules. These relationships are expressed in the universal language of 
  mathematics—through equations, functions, and deterministic or stochastic rules. The dynamical systems framework provide a powerful
  framework for modeling such behavior across disciplines, from physics and biology to economics and engineering.  

  Whether studying the motion of planets, the spread of diseases, neural activity, or climate patterns, dynamical systems reveal how 
  complex behavior emerges from simple interactions. By analyzing differential equations, iterative maps, or network dynamics, 
  we uncover patterns, stability, chaos, and predictability in nature. The multidisciplinary nature of dynamical systems highlights 
  the unity of scientific inquiry, demonstrating how mathematical principles underlie diverse real-world phenomena.
  \section{Dynamical Systems}

    \begin{definition}
      A dynamical system can be described as a collection or set of time dependent state variables $\mathbf{S}$
      $$\mathbf{S(t)} = \{ X_1(t), X_2(t), ..., X_n(t) \}$$
      
      that evolves with time according to a \textit{specific known deterministic rule} $\mathbf{f}$, such that
      
      $$S(t) \xrightarrow{\mathbf{f}} S(t + \tau)$$
    \end{definition}
    This \textbf{deterministic} rule $\mathbf{f}$ corresponds to a set of operations or procedures that allows to calculate the state
    of a system at time $t + \tau$ from the knolwdge of its state at time $t$.




  
  \sn{With some additional sidenotes}



  \end{document}